{
\tikzstyle{branch}=[fill,shape=circle,minimum size=2pt,inner sep=0pt]
	


\pgfdeclareshape{stabilizator}{
	\anchor{center}{\pgfpointorigin} % within the node, (0,0) is the center
	\anchor{text} % this is used to center the text in the node
	{  \pgfpoint{-.5\wd\pgfnodeparttextbox}{-.2\ht\pgfnodeparttextbox}  }
	
	\savedanchor\pina{\pgfpoint{-10mm}{-4mm}} 
	\anchor{vccIn}{\pina}
	\savedanchor\pinb{\pgfpoint{10mm}{-4mm}}
	\anchor{vOut}{\pinb}
	\savedanchor\pinc{\pgfpoint{-10mm}{4mm}}
	\anchor{gnd}{\pinc}
    \savedanchor\pind{\pgfpoint{10mm}{4mm}}
	\anchor{gndOut}{\pind}
	\foregroundpath{ % border and pin numbers are drawn here
		\pgfsetlinewidth{0.5mm}
		\pgfpathrectanglecorners{\pgfpoint{-9mm}{7mm}}{\pgfpoint{9mm}{-7mm}}
		
		\pgfusepath{draw} %draw rectangle
        \pgftext[at={\pgfpoint{0mm}{0mm}}]{\scriptsize TPS63020}
		\pgftext[left,at={\pgfpoint{-8mm}{-4mm}}]{\tiny VIN}
		\pgftext[right,at={\pgfpoint{8mm}{-4mm}}]{\tiny $+3{,}3$ В}
		\pgftext[left,at={\pgfpoint{-8mm}{4mm}}]{\tiny GND}
        \pgftext[right,at={\pgfpoint{8mm}{4mm}}]{\tiny GND}
	}
	\behindbackgroundpath{ %
		\path[draw, red, line width=0.5mm]   (-9mm, -4mm) -- (-10mm, -4mm);
		\path[draw, red, line width=0.5mm](9mm,    -4mm) -- (10mm,    -4mm);
		\path[draw, black, line width=0.5mm] (-9mm,  4mm) -- (-10mm,  4mm);
        \path[draw, black, line width=0.5mm] (9mm,  4mm) -- (10mm,  4mm);
	}
}


\pgfdeclareshape{zariadka}{
	\anchor{center}{\pgfpointorigin} % within the node, (0,0) is the center
	\anchor{text} % this is used to center the text in the node
	{  \pgfpoint{-.5\wd\pgfnodeparttextbox}{-.2\ht\pgfnodeparttextbox}  }
	
	\savedanchor\pina{\pgfpoint{-10mm}{-4mm}} 
	\anchor{vccIn}{\pina}
	\savedanchor\pinb{\pgfpoint{10mm}{-4mm}}
	\anchor{Bplus}{\pinb}
	\savedanchor\pinc{\pgfpoint{-10mm}{4mm}}
	\anchor{gnd}{\pinc}
    \savedanchor\pind{\pgfpoint{10mm}{4mm}}
	\anchor{Bminus}{\pind}
	\foregroundpath{ % border and pin numbers are drawn here
		\pgfsetlinewidth{0.5mm}
		\pgfpathrectanglecorners{\pgfpoint{-9mm}{7mm}}{\pgfpoint{9mm}{-7mm}}
		
		\pgfusepath{draw} %draw rectangle
        \pgftext[at={\pgfpoint{0mm}{0mm}}]{\scriptsize TP4056}
		\pgftext[left,at={\pgfpoint{-8mm}{-4mm}}]{\tiny $+5$ В}
		\pgftext[right,at={\pgfpoint{8mm}{-4mm}}]{\tiny Б $+$}
		\pgftext[left,at={\pgfpoint{-8mm}{4mm}}]{\tiny $-$}
        \pgftext[right,at={\pgfpoint{8mm}{4mm}}]{\tiny Б $-$}
	}
	\behindbackgroundpath{ %
		\path[draw, red, line width=0.5mm]   (-9mm, -4mm) -- (-10mm, -4mm);
		\path[draw, red, line width=0.5mm](9mm,    -4mm) -- (10mm,    -4mm);
		\path[draw, black, line width=0.5mm] (-9mm,  4mm) -- (-10mm,  4mm);
        \path[draw, black, line width=0.5mm] (9mm,  4mm) -- (10mm,  4mm);
	}
}


\pgfdeclareshape{batareika}{
	\anchor{center}{\pgfpointorigin} % within the node, (0,0) is the center
	\anchor{text} % this is used to center the text in the node
	{  \pgfpoint{-.5\wd\pgfnodeparttextbox}{-.2\ht\pgfnodeparttextbox}  }
	
	\savedanchor\pina{\pgfpoint{-4mm}{16mm}} 
	\anchor{plus}{\pina}
	\savedanchor\pinb{\pgfpoint{4mm}{16mm}}
	\anchor{gnd}{\pinb}
	\foregroundpath{ % border and pin numbers are drawn here
		\pgfsetlinewidth{0.5mm}
		\pgfpathrectanglecorners{\pgfpoint{-10mm}{15mm}}{\pgfpoint{10mm}{-15mm}}
		
		\pgfusepath{draw} %draw rectangle

        \pgftext[bottom,at={\pgfpoint{0mm}{5mm}}]{\tiny Литий-полимер}
        \pgftext[bottom,at={\pgfpoint{0mm}{3mm}}]{\tiny Номинал: $3{,}7$ В}
        \pgftext[bottom,at={\pgfpoint{0mm}{1mm}}]{\tiny Макс.: $4{,}2$ В}
		\pgftext[bottom,at={\pgfpoint{-4mm}{11mm}}]{$+$}
		\pgftext[bottom,at={\pgfpoint{4mm}{11mm}}]{$-$}
	}
	\behindbackgroundpath{ %
		\path[draw, red, line width=0.5mm]   (-4mm, 15mm) -- (-4mm, 16mm);
		\path[draw, black, line width=0.5mm] (4mm,  15mm) -- (4mm,  16mm);
	}
}



\begin{tikzpicture}[circuit ee IEC]

    \node[batareika, label={[shift={(0, 2cm)} , font=\scriptsize]Аккумулятор}] at (0, 0) (Akkumulator){};
	
	\node[stabilizator,  label={[shift={(0, 0.8cm)}, text  width=2.4cm, align=center, font=\scriptsize]Стабилизатор напряжения}] at ($(Akkumulator) + (3, 0.5)$) (Stabilizator){};

    \node[zariadka,  label={[shift={(0, 0.8cm)}, text  width=2.4cm, align=center, font=\scriptsize]Модуль зарядки}] at ($(Akkumulator) + (-3, 0.5)$) (Zariadka){};

    \draw[red] (Zariadka.Bplus) -- ++ (5mm, 0) |-  ($(Akkumulator.plus) + (0, 2mm)$) node[branch, red] (uselBplus) {} -- (Akkumulator.plus) {};
    
    \draw[black] (Zariadka.Bminus) -- ++ (2mm, 0) |- ($(Akkumulator.gnd) + (0, 4mm)$) node[branch, black](uselBminus) {} --  (Akkumulator.gnd) {};

    \draw[red] (Stabilizator.vccIn) -- ++ (-5mm, 0) |-  (uselBplus) {};

    \draw[black] (Stabilizator.gnd) -- ++ (-2mm, 0) |-  (uselBminus) {};

    \draw[red, -o] (Zariadka.vccIn) -- ++ (-7mm, 0) coordinate (USBplus) {};    
    \draw[black, -o] (Zariadka.gnd) -- ++ (-7mm, 0) coordinate (USBminus) {};

    \node[] at ($(USBplus)!0.5!(USBminus)$) (USB){USB};


    \draw[red, -o] (Stabilizator.vOut) -- ++ (7mm, 0) coordinate (Outplus) {};    
    \draw[black, -o] (Stabilizator.gndOut) -- ++ (7mm, 0) coordinate (Outminus) {};

    \node[align=center, font=\scriptsize] at ($(Outplus)!0.5!(Outminus)$) (Out){Питание\\ прибора};

    \node  at ($(Outplus)+(0.2, -0.2)$) (tmp) {\tiny 1};
    \node  at ($(Outminus)+(0.2, 0.2)$) (tmp) {\tiny 2};
    
\end{tikzpicture}

}